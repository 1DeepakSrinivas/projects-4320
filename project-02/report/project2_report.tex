\documentclass[12pt]{article}
\usepackage[margin=0.8in]{geometry}
\usepackage{graphicx}
\usepackage{amsmath}
\usepackage{fancyvrb}
\usepackage{listings}
\usepackage{xcolor}
\usepackage{hyperref}
\usepackage{multicol}

\lstset{
    basicstyle=\ttfamily\footnotesize,
    frame=single,
    breaklines=true,
    showstringspaces=false,
    backgroundcolor=\color{gray!10}
}

\title{
    \textbf{CSC 4320 Operating Systems} \\
    \large Project 02: Readers Writers Synchronization Problem
}
å
\author{Deepak Srinivas Govindarajan}
\date{November 15, 2025}

\begin{document}

\maketitle

\section{Overview}

This project implements the Readers Writers synchronization problem using POSIX threads. The design supports concurrent reading, exclusive writing, and limits writer starvation.\\
\textbf{Repository:} \url{https://github.com/1DeepakSrinivas/projects-4320}

\section{Implementation Details}

\subsection{Synchronization Primitives}

\textbf{Locks and Condition Variable:}
\begin{itemize}
    \item \texttt{count\_mutex}: protects reader counter \texttt{r\_count}.
    \item \texttt{mutex}: ensures exclusive writer access.
    \item \texttt{w\_cond}: coordinates waiting writers.
\end{itemize}

\textbf{Shared State:}
\begin{itemize}
    \item \texttt{shared\_data}: integer critical section value.
    \item \texttt{r\_count}: number of active readers.
    \item \texttt{w\_waiting}: number of writers queued.
\end{itemize}

\subsection{Algorithm Design}

Readers increment \texttt{r\_count} under \texttt{count\_mutex}. The first reader locks \texttt{mutex}, blocking writers. Subsequent readers enter concurrently. The last reader unlocks \texttt{mutex} and signals waiting writers.

Writers increment \texttt{w\_waiting}, then wait on \texttt{w\_cond} until \texttt{r\_count == 0}. Once admitted, a writer locks \texttt{mutex}, updates \texttt{shared\_data}, finishes, and signals waiting threads.

This enforces concurrent reads, exclusive writes, and limits writer starvation.

\subsection{Thread Configuration}

\begin{itemize}
    \item \texttt{NUM\_READERS = 5}
    \item \texttt{NUM\_WRITERS = 3}
    \item Reading time: 2 s
    \item Writing time: 3 s
\end{itemize}

Readers perform three read cycles. Writers perform two write cycles. Thread arguments include thread ID and type.

\subsection{Code Architecture}

The file \texttt{readers\_writers.c} implements:
\begin{itemize}
    \item \texttt{start\_read()}, \texttt{end\_read()}
    \item \texttt{start\_write()}, \texttt{end\_write()}
    \item \texttt{reader()} and \texttt{writer()} thread routines
\end{itemize}

The main control function creates and joins all threads, then destroys all synchronization primitives.

\section{Results}

\textbf{Concurrent Reading:} Multiple readers access \texttt{shared\_data} simultaneously. Logs confirm identical values for overlapping read intervals.

\textbf{Exclusive Writing:} Writers obtain exclusive access using \texttt{mutex}. Readers pause until writer completion. \texttt{shared\_data} increments sequentially with no race conditions.

\textbf{Lock Coordination:} Logs include thread creation, waiting, acquisition, critical section activity, and lock release. Execution shows no deadlocks or starvation.

\section{Files}

\texttt{output.txt} and \texttt{processes.txt} are included in \texttt{project-02/src}.

\section{Conclusion}

The project demonstrates proper synchronization for the Readers Writers problem using pthread mutexes and condition variables. The modular structure supports concurrent reads, exclusive writes, and fairness for writers.

\end{document}
